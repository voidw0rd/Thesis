\chapter{Introduction}
\section{Definition of Cloud computing}
Most people are confuse as to exactly what the term \emph{cloud computing} means, especially as the term can be used to mean almost anything.
Short story \emph{cloud computing} describes \emph{highly scalable computing resources} provided as an external service via the internet
on a pay-as-you-go basis. The \emph{cloud} is simply a metaphor for the internet, based on the symbol used to represent the worldwide
network in computer network diagrams.\\[0.2cm]
Cloud computing as a concept isn't nearly as new as most people think. In the past scientists tryed to approach this \emph{problem} but 
they failed because of insufficient processor performance, enormous hardware costs and slow bandwidth connections.\\[0.2cm]
However, today's technology, broadband internet connections and fast, cheap hardware, provide the opportunity to access only the services
and storage space that are actually necessary, and the ability to adjust these to meet current needs. Using a virtual server which is 
provided by a service provider, introduces a wide range of possibilities for cost savings, improved performance and higher data security.
Economically, the main appeal of \emph{cloud computing} is that customers only use what they need, and pay for what they actually use.
Resources are available to be accessed from the cloud at any time, and from any location via the internet. There is no need to worry 
about how things are being \emph{done} behind the scenes you simply buy the IT service you require as you would any other utility.
Because of this advantages, \emph{cloud computing} has been called in the past '\emph{outsourcing}' and '\emph{server hosting}' or 
'\emph{IT on demand}'.\\[0.2cm]
This new, web-based generation of computing utilises remote servers hosted in highly secure data centres for data storage and management,
so small or big organisations no longer need to spend lots of money to look after their IT solutions in-house.\\[0.1cm]

\clearpage
\section{Delivery Models}
The \emph{National Institute of Standards and Technology} definition of \emph{cloud computing} defines three delivery models:
\begin{itemize}
  \item {\bf Software as a Service (SaaS):} The consumer uses an application, but does not control the operating system, hardware or 
        network infrastructure on which it's running.
  \item {\bf Platform as a Service (PaaS):} The consumer uses a hosting environment for their applications. The consumer controls the 
        applications that run in the environment (and possibly has some control over the hosting environment), but does not control the 
        operating system, hardware or network infrastructure on which they are running. The platform is typically an application framework.        
  \item {\bf Infrastructure as a Service (IaaS):} The consumer uses "fundamental computing resources" such as processing power, storage, 
        networking components or middleware. The consumer can control the operating system, storage, deployed applications and possibly 
        networking components such as firewalls and load balancers, but not the cloud infrastructure beneath them.
\end{itemize}
\footnotetext{You can find the full document on the NIST Cloud Computing page at http://csrc.nist.gov/groups/SNS/cloud-computing/}

\section{Deployment Models}
The \emph{National Institute of Standards and Technology} defines four deployment models:
\begin{itemize}
  \item {\bf Public Cloud:} In simple terms, public cloud services are characterized as being available to clients from a third party 
        service provider via the Internet. The term “public” does not always mean free, even though it can be free or fairly inexpensive 
        to use. A public cloud does not mean that a user’s data is publically visible; public cloud vendors typically provide an access 
        control mechanism for their users. Public clouds provide an elastic, cost effective means to deploy solutions.
  \item {\bf Private Cloud:} A private cloud offers many of the benefits of a public cloud computing environment, such as being elastic 
        and service based. The difference between a private cloud and a public cloud is that in a private cloud-based service, data and
        processes are managed within the organization without the restrictions of network bandwidth, security exposures and legal 
        requirements that using public cloud services might entail. In addition, private cloud services offer the provider and the user 
        greater control of the cloud infrastructure, improving security and resiliency because user access and the networks used are 
        restricted and designated.
  \item {\bf Community Cloud:} A community cloud is controlled and used by a group of organizations that have shared interests, such 
        as specific security requirements or a common mission. The members of the community share access to the data and applications 
        in the cloud.
  \item {\bf Hybrid Cloud:} A hybrid cloud is a combination of a public and private cloud that interoperates. In this model users 
        typically outsource non-businesscritical information and processing to the public cloud, while keeping business-critical 
        services and data in their control.
\end{itemize}
\section{Essential Characteristics}
The \emph{National Institute of Standards and Technology} definition describes five essential characteristics of cloud computing:
\begin{itemize}
 \item {\bf Rapid Elasticity:} Elasticity is defined as the ability to scale resources both up and down as needed. To the consumer, 
       the cloud appears to be infinite, and the consumer can purchase as much or as little computing power as they need. This is one 
       of the essential characteristics of cloud computing in the NIST definition.
 \item {\bf Measured Service:} In a measured service, aspects of the cloud service are controlled and monitored by the cloud provider.
       This is crucial for billing, access control, resource optimization, capacity planning and other tasks. 
 \item {\bf On-Demand Self-Service:} The on-demand and self-service aspects of cloud computing mean that a consumer can use cloud 
       services as needed without any human interaction with the cloud provider. 
 \item {\bf Ubiquitous Network Access:} Ubiquitous network access means that the cloud provider’s capabilities are available over 
       the network and can be accessed through standard mechanisms by both thick and thin clients.
 \item {\bf Resource Pooling:} Resource pooling allows a  cloud provider to serve its consumers via a multi-tenant model. Physical 
       and virtual resources are assigned and reassigned according to consumer demand. There is a sense of location independence 
       in that the customer generally has no control or knowledge over the exact location of the provided resources but may be 
       able to specify location at a higher level of abstraction (e.g., country, state, or datacenter).
\end{itemize}
\footnotetext{You can find the full document on the NIST Cloud Computing page at http://csrc.nist.gov/groups/SNS/cloud-computing/}
\section{Other Terms}
\begin{itemize}
  \item {\bf Interoperability:} Interoperability is concerned with the ability of systems to communicate. It requires that the 
        communicated information is understood by the receiving system. In the world of cloud computing, this means the ability 
        to write code that works with more than one cloud provider simultaneously, regardless of the differences between the providers.
  \item {\bf Portability:} Portability is the ability to run components or systems written for one environment in another environment. 
        In the world of cloud computing, this includes software and hardware environments (both physical and virtual).
  \item {\bf Integration:} Integration is the process of combining components or systems into an overall system. Integration among 
        cloud-based components and systems can be complicated by issues such as multi-tenancy, federation and government regulations.        
  \item {\bf Service Level Agreement (SLA):} An SLA is contract between a provider and a consumer that specifies consumer requirements 
        and the provider’s commitment to them. Typically an SLA includes items such as uptime, privacy, security and backup procedures.
  \item {\bf Federation:} Federation is the act of combining data or identities across multiple systems. Federation can be done by a 
        cloud provider or by a cloud broker.
  \item {\bf Broker:} A broker has no cloud resources of its own, but matches consumers and providers based on the SLA required by the
        consumer. The consumer has no knowledge that the broker does not control the resources.      
  \item {\bf Multi-Tenancy:} Multi-tenancy is the property of multiple systems, applications or data from different enterprises hosted 
        on the same physical hardware. Multitenancy is common to most cloud-based systems.
  \item {\bf Cloud bursting:} Cloud bursting is a technique used by hybrid clouds to provide additional resources to private clouds on 
        an as-needed basis. If the private cloud has the processing power to handle its workloads, the hybrid cloud is not used.
        When workloads exceed the private cloud’s capacity, the hybrid cloud automatically allocates additional resources to the 
        private cloud.
  \item {\bf Policy:} A policy is a general term for an operating procedure. For example, a security policy might specify that all 
        requests to a particular cloud service must be encrypted.
  \item {\bf Governance:} Governance refers to the controls and processes that make sure policies are enforced.       
  \item {\bf Virtual Machine (VM):} A file (typically called an image) that, when executed, looks to the user like an actual machine. 
        Infrastructure as a Service is often provided as a VM image that can be started or stopped as needed. Changes made to the 
        VM while it is running can be stored to disk to make them persistent.        
\end{itemize}
\footnotetext{You can find the full document on the NIST Cloud Computing page at http://csrc.nist.gov/groups/SNS/cloud-computing/}
\section{What does it comprise?}
\emph{Cloud Computing} can be visualised as a \emph{pyramid} consisting of three sections:
\begin{itemize}
  \item {\bf Cloud Application} This is at the top of the \emph{pyramid}, where applications are running and you can interact with
        them via a web browser, hosted desktops or remote clients. Users never need to buy expensive software licenses themselves.
        Instead, the cost is incorporated into the subscription fee. A cloud application eliminates the need to install and run
        the application on the developer's own computer, thus removing the burden of software maintenance, ongoing operation and 
        support.
  \item {\bf Cloud Platform} This middle layer of the \emph{cloud computing pyramid}, which provides a computing platform or
        framework as a service. A cloud computing platform dynamically provisions, configures, reconfigures and de-provisions servers
        as needed to cope with increases or decreases in demand. This in reality is a distribured cumputing model, where many 
        services pull together to deliver and application or infrastructure request.
  \item {\bf Cloud Infrastructure} This bottom layer is the delivery of IT infrastructure through virtualization. Virtualization
        allows the splitting of a single physical machine into independent, self governed environments, which can be scaled in
        terms of CPU, RAM, Disk and other elements. The infrastructure includes servers, networks and other hardware delivered as either
        \emph{Infrastructure "Web Services", "Farms" or "cloud centres"}.
\end{itemize}
\section{What services can be used in the cloud?}
There are numerous services that can be delivered through cloud computing, taking advantage of thedistributed cloud model. Here are 
some brief descriptions of a few of the most popular cloud-based IT solutions: \\[0.2cm]
{\bf Hosted Desktops} \textendash\ Hosted desktops remove the need for traditional desktop PCs in the office environment, and reduce 
the cost of providing the services that you need. A hosted desktop looks and behaves like a regular desktop PC, but the 
software and data customers use are housed in remote, highly secure data centres, rather than on their own
machines. Users can simply access their hosted desktops via an internet connection from anywhere in the 
world, using either an existing PC or laptop or, for maximum cost efficiency, a specialised device called a thin 
client. \\[0.2cm]
{\bf Hosted Email} \textendash\ As more organisations look for a secure, reliable email solution that will not cost the earth, they are 
increasingly turning to hosted Microsoft Exchange® email plans. Using the world’s premier email platform, this 
service lets organisations both large and small reap the benefits of using MS Exchange® accounts without 
having to invest in the costly infrastructure themselves. Email is stored centrally on managed servers, 
providing redundancy and fast connectivity from any location. This allows users to access their email, calendar, 
contacts and shared files by a variety of means, including Outlook®, Outlook Mobile Access (OMA) and Outlook 
Web Access (OWA). \\[0.2cm]
{\bf Hosted Telephony (VOIP)} \textendash\ VOIP (Voice Over IP) is a means of carrying phone calls and services across digital internet networks. In terms of basic usage and functionality, VOIP is no different to traditional telephony, and a VOIP-enabled telephone 
works exactly like a 'normal' one, but it has distinct cost advantages. A hosted VOIP system replaces expensive 
phone systems, installation, handsets, BT lines and numbers with a simple, cost-efficient alternative that is
available to use on a monthly subscription basis. Typically, a pre-configured handset just needs to be plugged 
into your broadband or office network to allow you to access features such as voicemail, IVR and more. \\[0.2cm]
{\bf Cloud Storage} \textendash\ Cloud storage is growing in popularity due to the benefits it provides, such as simple, CapEx-free costs, 
anywhere access and the removal of the burden of in-house maintenance and management. It is basically the 
delivery of data storage as a service, from a third party provider, with access via the internet and billing 
calculated on capacity used in a certain period (e.g. per month).  
{\bf Dynamic Servers} \textendash\ Dynamic servers are the next generation of server environment, replacing the conventional concept of the 
dedicated server. A provider like ThinkGrid gives its customers access to resources that look and feel exactly 
like a dedicated server, but that are fully scalable. You can directly control the amount of processing power 
and space you use, meaning you don't have to pay for hardware you don't need. Typically, you can make 
changes to your dynamic server at any time, on the fly, without the costs associated with moving from one 
server to another.
\section{Cloud security}
Many developers that are thinking to move to \emph{the cloud} raise concerns over the security of data from the cloud how its stored and 
accessed via the internet. What a lot of them don't realise is that top cloud vendors adhere to strict privacy policies and sophisticated 
security measures, with data encryption being one example.
Companies can chose to encrypt data before even storing it on third-party provider's servers. As a result, many cloud computing vendors offer
greater data security and confidentiality than companies that choose to store their data in-house. 
Here is a list of security issues to bear in mind when considering moving to the cloud: 
\begin{itemize}
 \item Privileged user access \textendash\ enquire about who has access to data and about the hiring and management of such administrators
 \item Regulatory compliance \textendash\ make sure a vendor is willing to undergo external audits and/or security  certifications
 \item Data location \textendash\ ask if a provider allows for any control over the location of data
 \item Data segregation \textendash\ make sure that encryption is available at all stages and that these "encryption schemes were designed 
 and tested by experienced professionals"
 \item Recovery \textendash\ find out what will happen to data in the case of a disaster; do they offer complete restoration and, if so, how 
 long that would take
 \item Investigative Support \textendash\ inquire whether a vendor has the ability to investigate any inappropriate or illegal activity
 \item Long-term viability \textendash\ ask what will happen to data if the company goes out of business; how will data be returned and in what format 
\end{itemize} 
However security is usually improved by keeping data in one centralised location, in high security data centres.