\chapter{High-Resolution Monitoring}
\section{Measuring performance}
Performance measuring is done for two main reasons:\\
Firstly, we need to have some figures for marketing our system, secondly we need a diagnostic tool to aid us in finding performance
problems and bugs in our system during development process, it is nice to have information about the behaviour of the system \textendash\ 
Are there any peaks in the latency? If so how often? How high? What's the performance with single-core processors? What's the performance
with multi-core processors? \\[0.3cm]
As for the diagnostics for development process, a different aproach is needed and a different kind of performance measuring is done.
Instead of computing aggregate metrics we are more interested in the whole latency and throughout curves, we are not interested if there are 
peaks in performance, we are interested in where exacly these peaks occur. At the beginning of the test? When the load increases? Etc.\\[0.3cm]
Programmers employ a wide range of tools to help harden their code before it is deployed \textendash\ ranging from profiles to model checkers 
to test suites \textendash\ but this is not sufficient, this won't uncover software bugs, intermittent hardware failures, configuration errors
and unanticipared workloads of live production systems.
\clearpage
A set of problems that live production system face:
\begin{itemize}
  \item {\bf Comprehensiveness} \textendash\ enough data must be available to not only answer the question \emph{is the system healthy} but
  to also understand why it is not. This data doesn't need to be restricted to a set of detectable failures but should enable the diagnosis of 
  faults that cannot be characterized a priori. Data must be collected for the whole system to capture \emph{domino effects} that are typical 
  in build from componets systems.
  \item {\bf Robustness} \textendash\ diagnosis myust be possible even in conditions in which the system crashes, preventing postmortem 
  analysis of the live system, it should not be restricted to deterministic problems and be able to handle complex scenarios.
  \item {\bf Non-intrusiveness} \textendash\ system monitoring and diagnosis must not impose prohibitive overhead.
\end{itemize}
Using post-processing, aggregation and visualization tools, complex misbehaviors can be reconstructed through the quantitative and 
qualitative analysis of logs. For example, seemingly unresponsive services can be investigated by \emph{quantitatively} analyzing 
scheduling events to threshild total scheduling delays, which may be due to resource blocking or other reasons. Important events can then 
be studied  \emph{qualitatively} to isolate their root cause.
\section{Performance in the VM Environment}
The main advantages of the virtual machines are that multiple guest operating systems can co-exist on the same computer and it is in 
strong isolation from  each other.\\
Another important benefit of the virtual machines is to provide an instruction set architecture (ISA) that could be different from that 
of the real machine. Applications of virtual machines arise widely in computer science, such as in embedded systems, which support a 
real-time operating system at the same time as a high-level OS. In virtual machines(VMs) environment, the software layer which provides 
the virtualization is called a  virtual machine monitor (VMM).
The main types of virtualization technologies are:
\begin{itemize}
  \item {\bf Full-virtualization} \textendash\ is a virtualization technique used to provide a complete simulation of the underlying hardware \textendash\
  including the full instruction set, input/output operations, interrupts, memory access, and whatever other elements are used by the software 
  that runs on the bare machine, and that is intended to run in a virtual machine.
  \item {\bf Para-virtualization} \textendash\ is a virtualization technique that presents a software interface to virtual machines that is similar but 
  not identical to that of the underlying hardware. The paravirtualization's management module (a hypervisor or virtual machine monitor) operates with 
  an operating system that has been modified to work in a virtual machine.
  \item {\bf Pre-virtualization} \textendash\ is a new virtualization technique that instead of manually para-virtualising the host, or attempting to 
  rewrite binary code at load time it rewrites the assembly-language output of the compiler, a process called compiler afterburning.
\end{itemize}
Due to the modularity of the virtualization environments, VM has the following properties, such as flexible, secure, reliable, and easy to manage and 
configure, which can greatly minimized the hardware cost. However, besides these benefits, it also increases the complexity of virtual computing systems, 
and maximizes the performance of virtual computing systems.\\
Performance must be tuned for different types of application workloads. Tuning for large files and continuous performance isn't optimal for small files 
and vice versa, and tuning for high performance of small files isn't best for large files and continuous performance. If one or two VMs are running 
especially I/O-intensive applications and hammering a disk, other VMs that utilize the same data store may suffer the effects of disk I/O contention.
This problem is more difficult to identify than overloaded CPU or memory. Nowadays virtualization is ubiquitous and virtualization technologies play an 
important and significant role in many IT fields.
\subsection{Virtual Machine Monitor}
A virtual machine monitor (VMM) is a host program that allows a single computer to support multiple, identical execution environments. All the users see 
their systems as self-contained computers isolated from other users, even though every user is served by the same machine. In this context, a virtual 
machine is an operating system (OS) that is managed by an underlying control program. It can avoid any single program, running in one of the virtual machines, 
from overusing the resources of the host OS.\\
The software that creates a virtual machine environment in a computer. In a regular, nonvirtual environment, the operating system is the master control 
program, which manages the execution of all applications and acts as an interface between the applications and the hardware. In a virtual machine environment, 
the virtual machine monitor (VMM) becomes the master control program with the maximum privilege level, and the VMM manages one or more operating systems, 
now referred to as \emph{guest operating systems.} Each guest OS manages its own applications as it normally does in a non-virtual environment, except that 
it has been separated in the computer by the VMM. Each guest OS with its applications is known as a \emph{virtual machine} and is sometimes called a 
\emph{guest OS stack}.\\
In hardware virtualization, the concept host machine refers to the actual machine on which the virtualization takes place. The term guest machine, however, 
refers to the virtual machine. Likewise, the adjectives host and guest are used to help distinguish the software that run on the actual machine from those 
that run on the virtual machine. The software that creates a virtual machine on the host hardware is called Hypervisor or Virtual Machine Monitor.\\