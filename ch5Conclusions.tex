\chapter{Conclusions}
The long dreamed vision of computing as a utility is finally emerging. The elasticity of a utility matches the need of 
businesses providing services directly to customers over the Internet, as workloads can grow (and shrink) far faster. 
It used to take years to grow a business to several million customers \textendash\ now it can happen in months.\\[0.3cm]
From the cloud provider’s view, the construction of very large datacenters at low cost sites using commodity
computing, storage, and networking uncovered the possibility of selling those resources on a pay-as-you-go model
below the costs of many medium-sized datacenters, while making a profit by statistically multiplexing among a large
group of customers. From the cloud user's view, it would be as startling for a new software startup to build its own
datacenter as it would for a hardware startup to build its own fabrication line. In addition to startups, many other
established organizations take advantage of the elasticity of Cloud Computing regularly, including newspapers like the
Washington Post, movie companies like Pixar, and universities like ours. Our lab has benefited substantially from the
ability to complete research by conference deadlines and adjust resources over the semester to accommodate course
deadlines. As Cloud Computing users, we were relieved of dealing with the twin dangers of over-provisioning and
under-provisioning our internal datacenters.\\[0.3cm]
Some question whether companies accustomed to high-margin businesses, such as ad revenue from search engines
and traditional packaged software, can compete in Cloud Computing. First, the question presumes that Cloud Computing 
is a small margin business based on its low cost. Given the typical utilization of medium-sized datacenters, the
potential factors of 5 to 7 in economies of scale, and the further savings in selection of cloud datacenter locations, the
apparently low costs offered to cloud users may still be highly profitable to cloud providers. Second, these companies
may already have the datacenter, networking, and software infrastructure in place for their mainline businesses, so
Cloud Computing represents the opportunity for more income at little extra cost.
Although Cloud Computing providers may run afoul of the obstacles summarized in Table 6, we believe that over
the long run providers will successfully navigate these challenges and set an example for others to follow, perhaps by
successfully exploiting the opportunities that correspond to those obstacles.
Hence, developers would be wise to design their next generation of systems to be deployed into Cloud Computing. In 
general, the emphasis should be horizontal scalability to hundreds or thousands of virtual machines over the
efficiency of the system on a single virtual machine.\\
There are specific implications as well:
\begin{itemize}
  \item {\bf Applications Software} \textendash\ of the future will likely have a piece that runs on clients and a 
    piece that runs in the
    Cloud. The cloud piece needs to both scale down rapidly as well as scale up, which is a new requirement for
    software systems. The client piece needs to be useful when disconnected from the Cloud, which is not the case
    for many Web 2.0 applications today. Such software also needs a pay-for-use licensing model to match needs
    of Cloud Computing.
  \item {\bf Infrastructure Software} \textendash\ of the future needs to be cognizant that it is no longer running
    on bare metal but on virtual machines. Moreover, it needs to have billing built in from the beginning, as it is 
    very difficult to retrofit an accounting system.
  \item {\bf Hardware Systems} \textendash\ of the future need to be designed at the scale of a container (at 
    least a dozen racks) rather than at the scale of a single 1U box or single rack, as that is the minimum level at 
    which it will be purchased. Cost of operation will match performance and cost of purchase in importance in the 
    acquisition decision. Hence, they need to strive for energy proportionality [9] by making it possible to put 
    into low power mode the idle portions of the memory, storage, and networking, which already happens inside a 
    microprocessor today. Hardware should also be designed assuming that the lowest level software will be virtual 
    machines rather than a single native operating system, and it will need to facilitate flash as a new level of 
    the memory hierarchy between DRAM and disk. Finally, we need improvements in bandwidth and costs for both datacenter
     switches and WAN routers.
\end{itemize}
While im optimistic about the future of Cloud Computing, I would love to look into a crystal ball to see how popular 
it is and what it will look like in few years.\\
Will Cloud Computing be dominated by low-level hardware virtual machines like Amazon EC2, intermediate language offerings 
like Microsoft Azure, or high-level frameworks like Google AppEngine? Or will we have many virtualization levels that 
match different applications? Will value-added services by independent companies like RightScale, Heroku, or EngineYard 
survive in Utility Computing, or will the successful services be entirely co-opted by the Cloud providers? If they do 
consolidate to a single virtualization layer, will multiple companies embrace a common standard? Will this lead to a 
race to the bottom in pricing so that it’s unattractive to become a Cloud Computing provider, or will they differentiate 
in services or quality to maintain margins?