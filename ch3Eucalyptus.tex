\chapter{Eucalyptus Cloud Computing System}
\section{Introducing Eucalyptus}
{\bf Eucalyptus} \textendash\ an opensource software framework for cloud computing that implements what is commonly referred to as 
Infrastructure as a Service (IaaS); systems that give users the ability to run and control entire virtual machine instances deployed 
across a variety physical resources. We outline the basic principles of the Eucalyptus design, detail important operational aspects 
of the system, and discuss architectural trade-offs that we have made in order to allow Eucalyptus to be portable, modular and simple 
to use on infrastructure commonly found within academic settings.\\
Finally, we provide evidence that ucalyptus enables users familiar with existing Grid and HPC systems to explore new cloud computing 
functionality while maintaining access to existing, familiar application development software and Grid middle-ware.
\section{Eucalyptus design}
The architecture of the Eucalyptus system is simple, flexible and modular with a hierarchical design reflection common resource
environments found in many academic settings. In essence, the system allows users to start, control, access, and terminate entire 
virtual machines using an emulation of Amazon EC2's SOAP and "Query" interfaces. That is, users of Eucalyptus interact with the system 
using the exact same tools and interfaces that they use to interact with Amazon EC2. Currently, we support VMs that run atop the Xen 
hypervisor, but plan to add support for KVM/QEMU, VMware, and others in the near future.\\
We have chosen to implement each high-level system component as a stand-alone web service. This has the following benefits: first, each 
Web service exposes a welldefined language-agnostic API in the form of a WSDL document containing both operations that the service can 
perform and input/output data structures. Second, we can leverage existing Web-service features such as WSSecurity policies for secure 
communication between components. There are four high-level components, each with its own Web-service interface, that comprise a 
Eucalyptus installation:
\begin{itemize}
  \item {\bf Node Controller} \textendash\ controls the execution, inspection, and terminating of VM instances on the host 
  where it runs.
  \item {\bf Cluster Controller} \textendash\ gathers information about andschedules VM execution on specific node controllers, 
  as well as manages virtual instance network
  \item {\bf Storage Controller} \textendash\ (Walrus) is a put/get storage service that implements Amazon’s S3 interface, providing 
  a mechanism for storing and accessing virtual machine images and user data.
  \item {\bf Cloud Controller} \textendash\ is the entry-point into the cloud for users and administrators. It queries node managers 
  for information about resources, makes highlevel scheduling decisions, and implements them by making requests to cluster controllers.
\end{itemize}
\subsection{Node Controller}
An Node Controller (NC) executes on every node that is designated for hosting VM instances. An NC queries and controls the system 
software on its node (i.e., the host operating system and the hypervisor) in response to queries and control requests from its 
Cluster Controller.\\ 
An NC makes queries to discover the node’s physical resources – the number of cores, the size of memory, the available 
disk space – as well as to learn about the state of VM instances on the node (although an NC keeps track of the instances that it controls, 
instances may be started and stopped through mechanisms beyond NC’s control). The information thus collected is propagated up to the 
Cluster Controller in responses to describeResource and describeInstances requests.
\subsection{Cluster Controller}
The Cluster Controller (CC) generally executes on a cluster front-end machine, or any machine that has network connectivity to both the 
nodes running NCs and to the machine running the Cloud Controller (CLC). Many of the CC's operations are similar to the NC's operations 
but are generally plural instead of singular (e.g. runInstances, describeInstances, terminateInstances, describeResources). CC has 
three primary functions: schedule incoming instance run requests to specific NCs, control the instance virtual network overlay, and 
gather/report information about a set of NCs. When a CC receives a set of instances to run, it contacts each NC component through 
its describeResource operation and sends the runInstances request to the first NC that has enough free resources to host the instance. 
When a CC receives a describeResources request, it also receives a list of resource characteristics (cores, memory, and disk) 
describing the resource requirements needed by an instance (termed a VM "type"). With this information, the CC calculates how 
many simultaneous instances of the specific "type" can execute on its collection of NCs and reports that number back to the CLC.
\subsection{Storage Controller (Walrus)}
Eucalyptus includes Walrus, a data storage service that leverages standard web services technologies (Axis2, Mule) and is interface 
compatible with Amazon's Simple Storage Service (S3). Walrus implements the REST (via HTTP), sometimes termed the "Query" interface, 
as well as the SOAP interfaces that are compatible with S3.\\
Walrus provides two types of functionality:
\begin{itemize}
  \item Users that have access to Eucalyptus can use Walrus to stream data into/out of the cloud as well as from instances that they 
  have started on nodes.
  \item In addition, Walrus acts as a storage service for VM images. Root filesystem as well as kernel and ramdisk images used to 
  instantiate VMs on nodes can be uploaded to Walrus and accessed from nodes.
\end{itemize}
Users use standard S3 tools (either third party or those provided by Amazon) to stream data into and out of Walrus. The system shares 
user credentials with the Cloud Controller’s canonical user database.\\
Like S3, Walrus supports concurrent and serial data transfers. To aid scalability, Walrus does not provide locking for object writes. 
However, as is the case with S3, users are guaranteed that a consistent copy of the object will be saved if there are concurrent 
writes to the same object. If a write to an object is encountered while there is a previous write to the same object in progress, 
the previous write is invalidated. Walrus responds with the MD5 checksum of the object that was stored. Once a request has been verified, 
the user has been authenticated as a valid Eucalyptus user and checked against access control lists for the object that has been 
requested, writes and reads are streamed over HTTP.\\
Walrus also acts as an VM image storage and management service. VM root filesystem, kernel and ramdisk images are packaged and uploaded 
using standard EC2 tools provided by Amazon. These tools compress images, encrypt them using user credentials, and split them into 
multiple parts that are described in a image description file.\\ 
Walrus is entrusted with the task of verifying and decrypting images 
that have been uploaded by users. When a node controller (NC) requests an image from Walrus before instantiating it on a node, it 
sends an image download request that is authenticated using an internal set of credentials. Then, images are verified and decrypted, 
and finally transferred. As a performance optimization, and because VM images are often quite large, Walrus maintains a cache of images 
that have already been decrypted. Cache invalidation is done when an image manifest is overwritten, or periodically using a simple 
least recently used scheme. Walrus is designed to be modular such that the authentication, streaming and back-end storage 
subsystems can be customized by researchers to fit their needs.
\subsection{Cloud Controller}
The underlying virtualized resources that comprise a Eucalyptus cloud are exposed and managed by, the Cloud Controller (CLC). The CLC 
is a collection of webservices which are best grouped by their roles into three categories:
\begin{itemize}
  \item \emph{Resource Services} \textendash\ perform system-wide arbitration of resource allocations, let users manipulate properties 
  of the virtual machines and networks, and monitor both system components and virtual resources
  \item \emph{Data Services} \textendash\ govern persistent user and system data and provide for a configurable user environment for 
  formulating resource allocation request properties.
  \item \emph{Interface Services} \textendash\ present user visible interfaces, handling authentication and protocol translation, 
  and expose system management tools providing.
\end{itemize}
The Resource services process user virtual machine control requests and interact with the CCs to effect the allocation and deallocation 
of physical resources.\\
A simple representation of the system’s resource state (SRS) is maintained through communication with the CCs 
(as intermediates for interrogating the state of the NCs) and used in evaluating the realizability of user requests (vis a vis 
service-level agreements, or SLAs). The role of the SRS is executed in two stages: when user requests arrive, the information in the SRS 
is relied upon to make an admission control decision with respect to a user-specified service level expectation. VM creation, then, 
consists of reservation of the resources in the SRS, downstream request for VM creation, followed by commitment of the resources in the 
SRS on success, or rollback in case of errors.\\
The SRS then tracks the state of resource allocations and is the source of authority of changes to the properties of running reservations. 
SRS information is leveraged by a production rule system allowing for the formulation of an event-based SLA scheme. Application of an SLA is 
triggered by a corresponding event (e.g., network property changes, expiry of a timer) and can evaluate and modify the request (e.g., 
reject the request if it is unsatisfiable) or enact changes to the system state (e.g., time-limited allocations). While the system's 
representation in the SRS may not always reflect the actual resources, notably, the likelihood and nature of the inaccuracies can be 
quantified and considered when formulating and applying SLAs. Further, the admission control and the runtime SLA metrics work in conjunction 
to: ensure resources are not overcommitted and maintain a conservative view on resource availability to mitigate possibility of 
(service- level) failures.\\ 
The middle tier of Data Services handle the creation, modification, interrogation, and storage of stateful system and user data. Users can 
query these services to discover available resource information (images and clusters) and manipulate abstract parameters (keypairs, 
security groups, and network definitions) applicable to virtual machine and network allocations. The Resource Services interact with the 
Data Services to resolve references to user provided parameters (e.g., keys associated with a VM instance to be created). However, 
these services are not static configuration parameters. For example, a user is able to change, what amounts to, firewall rules which affect 
the ingress of traffic. The changes can be made offline and provided as inputs to a resource allocation request, but, additionally, they 
can be manipulated while the allocation is running. As a result, the services which manage the networking and security group data 
persistence must also act as agents of change on behalf of a user request to modify the state of a running collection of virtual 
machines and their supporting virtual network.
\section{Why use Eucalyptus?}
Well the eucalyptus system is built to allow administrators and researchers the ability to deploy an infrastructure for user-controlled 
virtual machine creation and control atop existing resources. Its hierarchical design targets resource commonly found within academic and 
laboratory settings, including but not limited to small- and mediumsized Linux clusters, workstation pools, and server farms.\\
Eucalyptus uses a virtual networking solution that provides VM isolation, high performance, and a view of the network that is simple and flat. 
The system is highly modular, with each module represented by a well-defined API, enabling researchers to replace components for 
experimentation with new cloud-computing solutions. Finally, the system exposes its feature set through a common user interface currently 
defined by Amazon EC2 and S3. This allows users who are familiar with EC2 and S3 to transition seamlessly to a Eucalyptus installation by, 
in most cases, a simple addition of a command-line argument or environment variable, instructing the client application where to send 
its messages.