\chapter{Cloud Kingdom \textendash\ \emph{application}}
This is a proof of concept on how to interact with the cloud, on how to use cloud resources and a simple monitoring system, 
the Cloud Kindom application follows the K.I.S.S. principle \emph{"Keep it simple stupid"}.
\section{Software solutions}
For this proof of concept project I decided to use the opensource cloud computing framework \emph{Eucalyptus}. Why ? Well 
\emph{Eucalyptus} is one of the most wide spread frameworks and you can already find a wide range of libraries and modules 
to aid you in \emph{talking} with eucalyptus.\\
My primary goal was to build an application that can start, stop, restart and access virtual machines from eucalyptus, to
build a simple monitoring system that can give me basic metrics like CPU usage, RAM usage, Disk usage, Network metrics and 
so on, and to make a web 2.0 interface that can use and display these features.\\
The application will be written (\emph{mostly}) in Python and Javascript, the following modules and libraries will be used:
\begin{itemize}
  \item {\bf Apache Libcloud} \textendash\ is a standard Python library that abstracts away differences among multiple 
  cloud provider APIs
  \item {\bf PSutil} \textendash\ is a module providing an interface for retrieving information on all running processes 
  and system utilization (CPU, disk, memory, network, users) in a portable way by using Python.
  \item {\bf jQuery} \textendash\ is a fast and concise JavaScript Library that simplifies HTML document traversing, 
  event handling, animating, and Ajax interactions for rapid web development.
  \item {\bf Highcharts} \textendash\ Highcharts is a charting library written in pure JavaScript, offering intuitive, 
  interactive charts to your web site or web application. Highcharts currently supports line, spline, area, areaspline, 
  column, bar, pie and scatter chart types.
  \item {\bf Socket.IO} \textendash\ aims to make realtime apps possible in every browser and mobile device, blurring 
  the differences between the different transport mechanisms. It's care-free realtime 100\% in JavaScript.
\end{itemize}
Most of these modules provide a simple interface that you can use and extend to fit your needs.
\subsection{Apache Libcloud}
Apache Libcloud is devided into 4 componets:
\begin{enumerate}
  \item {\bf Compute} \textendash\ allows you to manage cloud and virtual servers offered by different providers 
  such as Amazon, Rackspace, Linode and more.
  \begin{itemize}
    \item Node \textendash\ represents a cloud or virtual server.
    \item NodeSize \textendash\ represents node hardware configuration.
    \item NodeImage \textendash\ represents an operating system image.
    \item NodeLocation \textendash\ represents a server physical location.
    \item NodeState \textendash\ represents a node state:
    \begin{itemize}
      \item running 
      \item rebooting
      \item terminated
      \item pending 
      \item unknown
    \end{itemize}
  \end{itemize}
  \item {\bf Storage} \textendash\ API allows you to manage cloud storage and services such as Amazon S3, Rackspace 
  CloudFiles, Google Storage.
  \begin{itemize}
    \item Object \textendash\ represents an object or so called BLOB.
    \item Container \textendash\ represents a container which can contain multiple objects. Some APIs also call it a Bucket.
  \end{itemize}
  \item {\bf Load Balancer} \textendash\  allows you to manage Load Balancers as a service and services such as Rackspace 
  Cloud Load Balancers, GoGrid Load Balancers and Ninefold Load Balancers.
  \begin{itemize}
    \item LoadBalancer \textendash\ represents a load balancer instance.
    \item Member \textendash\ represents a load balancer member.
    \item Algorithm \textendash\ represents a load balancing algorithm (round-robin, random, etc).
  \end{itemize}
  \item {\bf DNS} \textendash\ allows you to manage DNS as A Service and services such as Zerigo DNS, Rackspace Cloud DNS.
  \begin{itemize}
    \item Zone \textendash\ represents a DNS zone or so called domain.
    \item Record \textendash\ represents a DNS record.
    \item RecordType \textendash\ represents a record type (A, AAAA, MX, TXT, etc.).
  \end{itemize}
\end{enumerate}
\subsubsection{Examples}
Deploying an EC2 Node using SSH key authentication
  \begin{verbatim}
  #!/bin/env python
  from libcloud.compute.types import Provider
  from libcloud.compute.providers import get_driver
  from libcloud.compute.deployment import ScriptDeployment

  EC2_ACCESS_ID = 'your access key'
  EC2_SECRET = 'your secret'
  Driver = get_driver(Provider.EC2_US_EAST)
  conn = Driver(EC2_ACCESS_ID, EC2_SECRET)
  image = [i for i in driver.list_images() if i.id =='ami-09965860' ][0]
  size = [s for s in driver.list_sizes() if s.id == 't1.micro'][0]
  key_name = 'my_default'
  key_path = '/home/user/path/to/key.pem'
  node = conn.deploy_node(name='test', image=image, size=size, deploy=script,
                          ssh_username='ubuntu', ssh_key=key_path,
                          ex_keyname=key_name)
  \end{verbatim}
\subsection{PSutil}
PSutil is a module providing an interface for retrieving information on all running processes and system utilization 
(CPU, disk, memory, network, users) in a portable way by using Python, implementing many functionalities offered by 
command line tools such as:
\hspace{0.5cm}ps, top, df, kill, free, lsof, netstat, ifconfig, ncie, ionice, iostat, iotop, pidof, uptime, tty, who, etc.
\subsubsection{Examples}
\begin{itemize}
  \item CPU \textendash\ get cpu times 
  \begin{verbatim}
    >>> import psutil
    >>> psutil.cpu_times()
    cputimes(user=3961.46, nice=169.729, 
             system=2150.659, idle=16900.540, 
             iowait=629.509, irq=0.0, softirq=19.422)
    \end{verbatim}
  \item Memory \textendash\ get memory usage
  \begin{verbatim}
    >>> import psutil
    >>> psutil.phymem_usage()
    usage(total=4153868288, used=2854199296, 
          free=1299668992, percent=34.6)
    >>> psutil.virtmem_usage()
    usage(total=2097147904, used=4096, 
          free=2097143808, percent=0.0)          
  \end{verbatim}
  \item Disk \textendash\ get disk information
  \begin{verbatim}
    >>> import psutil
    >>> psutil.disk_partitions()
    [partition(device='/dev/sda1', mountpoint='/', fstype='ext4'),
    partition(device='/dev/sda2', mountpoint='/home', fstype='ext4')]
    >>> psutil.disk_usage('/')
    usage(total=21378641920, used=4809781248, 
          free=15482871808, percent=22.5)
  \end{verbatim}
  \item Network \textendash\ get network information
  \begin{verbatim}
    >>> import psutil
    >>> psutil.network_io_counters(pernic=True)
    {'lo': iostat(bytes_sent=799953745, bytes_recv=799953745, 
     packets_sent=453698, packets_recv=453698), 
     'eth0': iostat(bytes_sent=734324837, bytes_recv=4163935363, 
     packets_sent=3605828, packets_recv=4096685)}
  \end{verbatim}
\end{itemize}
\subsection{jQuery}
jQuery is an opensource project and one of the most advanced and wide spread javascript library, jQuery is made 
by and for the community its fast and it simplifies HTML document traversing, event handling, animating, and 
Ajax interactions for rapid web development.\\
The jQuery library is CSS3 compliant its cross-browser and lightweight, its syntax is designed to make it easier
to work with the DOM.\\
jQuery also provides capabilities for developers to create plug-ins on top of the JavaScript library. 
This enables developers to create abstractions for low-level interaction and animation, advanced effects and 
high-level, theme-able widgets. The modular approach to the jQuery library allows the creation of powerful dynamic 
web pages and web applications.
\subsubsection{Examples}
Prevent the default behaviour of an anchor tag when clicked.
\begin{verbatim}
  $(document).ready(function(){
    $("a").click(function(event){
      alert("As you can see, the link no longer took you to jquery.com");
      $(this).hide();
      event.preventDefault();
    });
  });
\end{verbatim}
Make an element fade in slowly, and add a css class to it.
\begin{verbatim}
  $(document).ready(function(){
    $("#element").fadeIn("slow").addClass("active"); });
\end{verbatim}
Make an Ajax call.
\begin{verbatim}
  $(document).ready(function(){
    $.ajax({
      url: "test.html",
      context: document.body
    }).done(function() { 
      $(this).addClass("done");
    });
  });
\end{verbatim}
\subsection{HighCharts}
Highcharts is a charting library written in pure JavaScript, offering an easy way of adding interactive charts to 
your web site or web application. Highcharts currently supports line, spline, area, areaspline, column, bar, pie 
and scatter chart types.\\
It works in all modern browsers including the iPhone/iPad and Internet Explorer from version 6. Standard browsers 
use SVG for the graphics rendering. In legacy Internet Explorer graphics are drawn using VML.\\
Highcharts is solely based on native browser technologies and doesn't require client side plugins like Flash or Java. 
Furthermore you don't need to install anything on your server. No PHP or ASP.NET. Highcharts needs only two JS files 
to run: The highcharts.js core and either the jQuery, MooTools or Prototype framework. One of these frameworks is most 
likely already in use in your web page.\\
Setting the Highcharts configuration options requires no special programming skills. The options are given in a 
JavaScript object notation structure, which is basically a set of keys and values connected by colons, separated by 
commas and grouped by curly brackets.\\
Through a full API you can add, remove and modify series and points or modify axes at any time after chart creation. 
Numerous events supply hooks for programming agains the chart. In combination with jQuery, MooTools or Prototype's 
Ajax API, this opens for solutions like live charts constantly updating with values from the server, user supplied 
data and more.\\
\clearpage
\subsubsection{Examples}
Basic line Chart
\begin{verbatim}
chart = new Highcharts.Chart({
  chart: {
    renderTo: 'container',
    type: 'line',
    marginRight: 130,
    marginBottom: 25
  },
  xAxis: {
    categories: ['Jan', 'Feb', 'Mar', 'Apr', 'May', 'Jun',
                 'Jul', 'Aug', 'Sep', 'Oct', 'Nov', 'Dec']
  },
  yAxis: {
    title: {
      text: 'Temperature (°C)'
    },
    plotLines: [{
      value: 0,
      width: 1,
      color: '#808080'
    }]
  },
)};  
\end{verbatim}
\subsection{Socket.IO}
Socket.IO is a Node.JS project that makes WebSockets and realtime possible in all browsers. It also enhances 
WebSockets by providing built-in multiplexing, horizontal scalability, automatic JSON encoding/decoding, and more.
Socket.IO aims to make realtime apps possible in every browser and mobile device, blurring the differences between 
the different transport mechanisms. It's care-free realtime 100\% in JavaScript.
\clearpage
\subsubsection{Examples}
Simple echo server using websockets.\\
{\bf Server side}
\begin{verbatim}
var io = require('socket.io').listen(80);

io.sockets.on('connection', function (socket) {
    socket.emit('news', { hello: 'world' });
    socket.on('my other event', function (data) {
    console.log(data);
  });
});
\end{verbatim}
{\bf Client side}
\begin{verbatim}
<script src="/socket.io/socket.io.js"></script>
<script>
  var socket = io.connect('http://localhost');
  
  socket.on('news', function (data) {
    console.log(data);
    socket.emit('my other event', { my: 'data' });
  });
</script>
\end{verbatim}
\section{Application Architecture}
The application is made up of two big componets, one is for consuming cloud resources like starting, stoping virtual 
machines and the other one is the lightweight monitoring system.
\begin{itemize}
  \item 
\end{itemize}